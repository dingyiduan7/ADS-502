% Options for packages loaded elsewhere
\PassOptionsToPackage{unicode}{hyperref}
\PassOptionsToPackage{hyphens}{url}
%
\documentclass[
]{article}
\usepackage{amsmath,amssymb}
\usepackage{lmodern}
\usepackage{ifxetex,ifluatex}
\ifnum 0\ifxetex 1\fi\ifluatex 1\fi=0 % if pdftex
  \usepackage[T1]{fontenc}
  \usepackage[utf8]{inputenc}
  \usepackage{textcomp} % provide euro and other symbols
\else % if luatex or xetex
  \usepackage{unicode-math}
  \defaultfontfeatures{Scale=MatchLowercase}
  \defaultfontfeatures[\rmfamily]{Ligatures=TeX,Scale=1}
\fi
% Use upquote if available, for straight quotes in verbatim environments
\IfFileExists{upquote.sty}{\usepackage{upquote}}{}
\IfFileExists{microtype.sty}{% use microtype if available
  \usepackage[]{microtype}
  \UseMicrotypeSet[protrusion]{basicmath} % disable protrusion for tt fonts
}{}
\makeatletter
\@ifundefined{KOMAClassName}{% if non-KOMA class
  \IfFileExists{parskip.sty}{%
    \usepackage{parskip}
  }{% else
    \setlength{\parindent}{0pt}
    \setlength{\parskip}{6pt plus 2pt minus 1pt}}
}{% if KOMA class
  \KOMAoptions{parskip=half}}
\makeatother
\usepackage{xcolor}
\IfFileExists{xurl.sty}{\usepackage{xurl}}{} % add URL line breaks if available
\IfFileExists{bookmark.sty}{\usepackage{bookmark}}{\usepackage{hyperref}}
\hypersetup{
  pdftitle={ADS502-Assignment-2.1-R.R},
  pdfauthor={DDY},
  hidelinks,
  pdfcreator={LaTeX via pandoc}}
\urlstyle{same} % disable monospaced font for URLs
\usepackage[margin=1in]{geometry}
\usepackage{color}
\usepackage{fancyvrb}
\newcommand{\VerbBar}{|}
\newcommand{\VERB}{\Verb[commandchars=\\\{\}]}
\DefineVerbatimEnvironment{Highlighting}{Verbatim}{commandchars=\\\{\}}
% Add ',fontsize=\small' for more characters per line
\usepackage{framed}
\definecolor{shadecolor}{RGB}{248,248,248}
\newenvironment{Shaded}{\begin{snugshade}}{\end{snugshade}}
\newcommand{\AlertTok}[1]{\textcolor[rgb]{0.94,0.16,0.16}{#1}}
\newcommand{\AnnotationTok}[1]{\textcolor[rgb]{0.56,0.35,0.01}{\textbf{\textit{#1}}}}
\newcommand{\AttributeTok}[1]{\textcolor[rgb]{0.77,0.63,0.00}{#1}}
\newcommand{\BaseNTok}[1]{\textcolor[rgb]{0.00,0.00,0.81}{#1}}
\newcommand{\BuiltInTok}[1]{#1}
\newcommand{\CharTok}[1]{\textcolor[rgb]{0.31,0.60,0.02}{#1}}
\newcommand{\CommentTok}[1]{\textcolor[rgb]{0.56,0.35,0.01}{\textit{#1}}}
\newcommand{\CommentVarTok}[1]{\textcolor[rgb]{0.56,0.35,0.01}{\textbf{\textit{#1}}}}
\newcommand{\ConstantTok}[1]{\textcolor[rgb]{0.00,0.00,0.00}{#1}}
\newcommand{\ControlFlowTok}[1]{\textcolor[rgb]{0.13,0.29,0.53}{\textbf{#1}}}
\newcommand{\DataTypeTok}[1]{\textcolor[rgb]{0.13,0.29,0.53}{#1}}
\newcommand{\DecValTok}[1]{\textcolor[rgb]{0.00,0.00,0.81}{#1}}
\newcommand{\DocumentationTok}[1]{\textcolor[rgb]{0.56,0.35,0.01}{\textbf{\textit{#1}}}}
\newcommand{\ErrorTok}[1]{\textcolor[rgb]{0.64,0.00,0.00}{\textbf{#1}}}
\newcommand{\ExtensionTok}[1]{#1}
\newcommand{\FloatTok}[1]{\textcolor[rgb]{0.00,0.00,0.81}{#1}}
\newcommand{\FunctionTok}[1]{\textcolor[rgb]{0.00,0.00,0.00}{#1}}
\newcommand{\ImportTok}[1]{#1}
\newcommand{\InformationTok}[1]{\textcolor[rgb]{0.56,0.35,0.01}{\textbf{\textit{#1}}}}
\newcommand{\KeywordTok}[1]{\textcolor[rgb]{0.13,0.29,0.53}{\textbf{#1}}}
\newcommand{\NormalTok}[1]{#1}
\newcommand{\OperatorTok}[1]{\textcolor[rgb]{0.81,0.36,0.00}{\textbf{#1}}}
\newcommand{\OtherTok}[1]{\textcolor[rgb]{0.56,0.35,0.01}{#1}}
\newcommand{\PreprocessorTok}[1]{\textcolor[rgb]{0.56,0.35,0.01}{\textit{#1}}}
\newcommand{\RegionMarkerTok}[1]{#1}
\newcommand{\SpecialCharTok}[1]{\textcolor[rgb]{0.00,0.00,0.00}{#1}}
\newcommand{\SpecialStringTok}[1]{\textcolor[rgb]{0.31,0.60,0.02}{#1}}
\newcommand{\StringTok}[1]{\textcolor[rgb]{0.31,0.60,0.02}{#1}}
\newcommand{\VariableTok}[1]{\textcolor[rgb]{0.00,0.00,0.00}{#1}}
\newcommand{\VerbatimStringTok}[1]{\textcolor[rgb]{0.31,0.60,0.02}{#1}}
\newcommand{\WarningTok}[1]{\textcolor[rgb]{0.56,0.35,0.01}{\textbf{\textit{#1}}}}
\usepackage{graphicx}
\makeatletter
\def\maxwidth{\ifdim\Gin@nat@width>\linewidth\linewidth\else\Gin@nat@width\fi}
\def\maxheight{\ifdim\Gin@nat@height>\textheight\textheight\else\Gin@nat@height\fi}
\makeatother
% Scale images if necessary, so that they will not overflow the page
% margins by default, and it is still possible to overwrite the defaults
% using explicit options in \includegraphics[width, height, ...]{}
\setkeys{Gin}{width=\maxwidth,height=\maxheight,keepaspectratio}
% Set default figure placement to htbp
\makeatletter
\def\fps@figure{htbp}
\makeatother
\setlength{\emergencystretch}{3em} % prevent overfull lines
\providecommand{\tightlist}{%
  \setlength{\itemsep}{0pt}\setlength{\parskip}{0pt}}
\setcounter{secnumdepth}{-\maxdimen} % remove section numbering
\ifluatex
  \usepackage{selnolig}  % disable illegal ligatures
\fi

\title{ADS502-Assignment-2.1-R.R}
\author{DDY}
\date{2021-07-11}

\begin{document}
\maketitle

\begin{Shaded}
\begin{Highlighting}[]
\CommentTok{\# Assignment 2.1 [R]}

\CommentTok{\# University of San Diego}

\CommentTok{\# ADS 502}

\CommentTok{\# Dingyi Duan}


\CommentTok{\# For Exercises 21{-}30, continue working with the bank\_marketing\_training }
\CommentTok{\# data set. Use either Python or R to solve each problem.}

\CommentTok{\# 21. Produce the following graphs. What is the strength of each graph? Weakness?}

\CommentTok{\# a. Bar graph of marital.}

\FunctionTok{library}\NormalTok{(ggplot2)}

\NormalTok{bank\_train }\OtherTok{\textless{}{-}} \FunctionTok{read.csv}\NormalTok{(}\AttributeTok{file =} \StringTok{"C:/Users/DDY/Desktop/2021{-}Spring{-}textbooks/ADS{-}502/Module2/Website Data Sets/bank\_marketing\_training.csv"}\NormalTok{)}

\FunctionTok{ggplot}\NormalTok{(bank\_train, }\FunctionTok{aes}\NormalTok{(marital)) }\SpecialCharTok{+} \FunctionTok{geom\_bar}\NormalTok{() }\SpecialCharTok{+} \FunctionTok{coord\_flip}\NormalTok{()}
\end{Highlighting}
\end{Shaded}

\includegraphics{ADS502-Assignment-2.1-R_files/figure-latex/unnamed-chunk-1-1.pdf}

\begin{Shaded}
\begin{Highlighting}[]
\CommentTok{\# b. Bar graph of marital, with overlay of response.}

\FunctionTok{ggplot}\NormalTok{(bank\_train, }\FunctionTok{aes}\NormalTok{(marital)) }\SpecialCharTok{+} \FunctionTok{geom\_bar}\NormalTok{(}\FunctionTok{aes}\NormalTok{(}\AttributeTok{fill =}\NormalTok{ response)) }\SpecialCharTok{+} \FunctionTok{coord\_flip}\NormalTok{()}
\end{Highlighting}
\end{Shaded}

\includegraphics{ADS502-Assignment-2.1-R_files/figure-latex/unnamed-chunk-1-2.pdf}

\begin{Shaded}
\begin{Highlighting}[]
\CommentTok{\# c. Normalized bar graph of marital, with overlay of response.}

\FunctionTok{ggplot}\NormalTok{(bank\_train, }\FunctionTok{aes}\NormalTok{(marital)) }\SpecialCharTok{+} \FunctionTok{geom\_bar}\NormalTok{(}\FunctionTok{aes}\NormalTok{(}\AttributeTok{fill =}\NormalTok{ response),}
                                            \AttributeTok{position =} \StringTok{"fill"}\NormalTok{) }\SpecialCharTok{+} \FunctionTok{coord\_flip}\NormalTok{()}
\end{Highlighting}
\end{Shaded}

\includegraphics{ADS502-Assignment-2.1-R_files/figure-latex/unnamed-chunk-1-3.pdf}

\begin{Shaded}
\begin{Highlighting}[]
\CommentTok{\# 22. Using the graph from Exercise 21c, describe the relationship between marital and response.}
\CommentTok{\# In divorced and married status, the response of "yes" rate is the same and the lowest among all;}
\CommentTok{\# For unknown status, the response of "yes" rate is in between single and divorced/married;}
\CommentTok{\# Response rate of "yes" is the highest for single marital status}


\DocumentationTok{\#\# 23. Do the following with the variables marital and response.}

\CommentTok{\# a. Build a contingency table, being careful to have the correct variables }
\CommentTok{\# representing the rows and columns. Report the counts and the column percentages.}

\NormalTok{t.v1 }\OtherTok{\textless{}{-}} \FunctionTok{table}\NormalTok{(bank\_train}\SpecialCharTok{$}\NormalTok{response, bank\_train}\SpecialCharTok{$}\NormalTok{marital)}
\NormalTok{t.v2 }\OtherTok{\textless{}{-}} \FunctionTok{addmargins}\NormalTok{(}\AttributeTok{A =}\NormalTok{ t.v1, }\AttributeTok{FUN =} \FunctionTok{list}\NormalTok{(}\AttributeTok{total =}\NormalTok{ sum),}\AttributeTok{quiet =} \ConstantTok{TRUE}\NormalTok{)}

\NormalTok{t.v1}
\end{Highlighting}
\end{Shaded}

\begin{verbatim}
##      
##       divorced married single unknown
##   no      2743   14579   6514      50
##   yes      312    1608   1061       7
\end{verbatim}

\begin{Shaded}
\begin{Highlighting}[]
\NormalTok{t.v2}
\end{Highlighting}
\end{Shaded}

\begin{verbatim}
##        
##         divorced married single unknown total
##   no        2743   14579   6514      50 23886
##   yes        312    1608   1061       7  2988
##   total     3055   16187   7575      57 26874
\end{verbatim}

\begin{Shaded}
\begin{Highlighting}[]
\NormalTok{t.v1\_pct }\OtherTok{\textless{}{-}} \FunctionTok{round}\NormalTok{(}\FunctionTok{prop.table}\NormalTok{(t.v1, }\AttributeTok{margin =} \DecValTok{2}\NormalTok{)}\SpecialCharTok{*}\DecValTok{100}\NormalTok{, }\DecValTok{1}\NormalTok{)}
\NormalTok{t.v2\_pct }\OtherTok{\textless{}{-}} \FunctionTok{addmargins}\NormalTok{(}\AttributeTok{A =}\NormalTok{ t.v1\_pct, }\AttributeTok{FUN =} \FunctionTok{list}\NormalTok{(}\AttributeTok{total =}\NormalTok{ sum),}\AttributeTok{quiet =} \ConstantTok{TRUE}\NormalTok{)}

\NormalTok{t.v1\_pct}
\end{Highlighting}
\end{Shaded}

\begin{verbatim}
##      
##       divorced married single unknown
##   no      89.8    90.1   86.0    87.7
##   yes     10.2     9.9   14.0    12.3
\end{verbatim}

\begin{Shaded}
\begin{Highlighting}[]
\CommentTok{\# b. Describe what the contingency table is telling you.}
\CommentTok{\# For response of "no", \textquotesingle{}married\textquotesingle{} has the most percentage;}
\CommentTok{\# For response of "yes", \textquotesingle{}single\textquotesingle{} has the most percentage.}

\CommentTok{\# 24. Repeat the previous exercise, this time reporting the row percentages. Explain the}
\CommentTok{\# difference between the interpretation of this table and the previous contingency table.}

\NormalTok{t.v1\_r }\OtherTok{\textless{}{-}} \FunctionTok{table}\NormalTok{(bank\_train}\SpecialCharTok{$}\NormalTok{marital, bank\_train}\SpecialCharTok{$}\NormalTok{response)}
\NormalTok{t.v2\_r }\OtherTok{\textless{}{-}} \FunctionTok{addmargins}\NormalTok{(}\AttributeTok{A =}\NormalTok{ t.v1\_r, }\AttributeTok{FUN =} \FunctionTok{list}\NormalTok{(}\AttributeTok{total =}\NormalTok{ sum),}\AttributeTok{quiet =} \ConstantTok{TRUE}\NormalTok{)}

\NormalTok{t.v1\_r}
\end{Highlighting}
\end{Shaded}

\begin{verbatim}
##           
##               no   yes
##   divorced  2743   312
##   married  14579  1608
##   single    6514  1061
##   unknown     50     7
\end{verbatim}

\begin{Shaded}
\begin{Highlighting}[]
\NormalTok{t.v2\_r}
\end{Highlighting}
\end{Shaded}

\begin{verbatim}
##           
##               no   yes total
##   divorced  2743   312  3055
##   married  14579  1608 16187
##   single    6514  1061  7575
##   unknown     50     7    57
##   total    23886  2988 26874
\end{verbatim}

\begin{Shaded}
\begin{Highlighting}[]
\NormalTok{t.v1\_r\_pct }\OtherTok{\textless{}{-}} \FunctionTok{round}\NormalTok{(}\FunctionTok{prop.table}\NormalTok{(t.v1\_r, }\AttributeTok{margin =} \DecValTok{1}\NormalTok{)}\SpecialCharTok{*}\DecValTok{100}\NormalTok{, }\DecValTok{1}\NormalTok{)}
\NormalTok{t.v2\_r\_pct }\OtherTok{\textless{}{-}} \FunctionTok{addmargins}\NormalTok{(}\AttributeTok{A =}\NormalTok{ t.v1\_r\_pct, }\AttributeTok{FUN =} \FunctionTok{list}\NormalTok{(}\AttributeTok{total =}\NormalTok{ sum),}\AttributeTok{quiet =} \ConstantTok{TRUE}\NormalTok{)}

\NormalTok{t.v1\_r\_pct}
\end{Highlighting}
\end{Shaded}

\begin{verbatim}
##           
##              no  yes
##   divorced 89.8 10.2
##   married  90.1  9.9
##   single   86.0 14.0
##   unknown  87.7 12.3
\end{verbatim}

\begin{Shaded}
\begin{Highlighting}[]
\CommentTok{\# This time the row percentage shows the ratio in each marital status of response of "yes" and "no";}
\CommentTok{\# In "divorced", 89.79\% responded "no" and 10.21\% responded "yes";}
\CommentTok{\# In "married", 90.07\% responded "no" and 9.93\% responded "yes";}
\CommentTok{\# In "single", 85.99\% responded "no" and 14.01\% responded "yes";}
\CommentTok{\# In "unknown", 87.72\% responded "no" and 12.38\% responded "yes";}
\CommentTok{\# Overall, more people recompensed "no" than "yes".}

\CommentTok{\# The difference between this two tables is one is from the perspective of }
\CommentTok{\# response while the other is }
\CommentTok{\# from the perspective of marital status.}


\DocumentationTok{\#\#\# 25. Produce the following graphs. What is the strength of each graph? Weakness?}

\CommentTok{\# a. Histogram of duration.}

\FunctionTok{ggplot}\NormalTok{(bank\_train, }\FunctionTok{aes}\NormalTok{(duration)) }\SpecialCharTok{+} \FunctionTok{geom\_histogram}\NormalTok{(}\AttributeTok{color=}\StringTok{"black"}\NormalTok{)}
\end{Highlighting}
\end{Shaded}

\begin{verbatim}
## `stat_bin()` using `bins = 30`. Pick better value with `binwidth`.
\end{verbatim}

\includegraphics{ADS502-Assignment-2.1-R_files/figure-latex/unnamed-chunk-1-4.pdf}

\begin{Shaded}
\begin{Highlighting}[]
\CommentTok{\# b. Histogram of duration, with overlay of response.}

\FunctionTok{ggplot}\NormalTok{(bank\_train, }\FunctionTok{aes}\NormalTok{(duration)) }\SpecialCharTok{+} \FunctionTok{geom\_histogram}\NormalTok{(}\FunctionTok{aes}\NormalTok{(}\AttributeTok{fill =}\NormalTok{ response), }\AttributeTok{color=}\StringTok{"black"}\NormalTok{)}
\end{Highlighting}
\end{Shaded}

\begin{verbatim}
## `stat_bin()` using `bins = 30`. Pick better value with `binwidth`.
\end{verbatim}

\includegraphics{ADS502-Assignment-2.1-R_files/figure-latex/unnamed-chunk-1-5.pdf}

\begin{Shaded}
\begin{Highlighting}[]
\CommentTok{\# c. Normalized histogram of duration, with overlay of response.}

\FunctionTok{ggplot}\NormalTok{(bank\_train, }\FunctionTok{aes}\NormalTok{(duration)) }\SpecialCharTok{+} \FunctionTok{geom\_histogram}\NormalTok{(}\FunctionTok{aes}\NormalTok{(}\AttributeTok{fill =}\NormalTok{ response), }\AttributeTok{color=}\StringTok{"black"}\NormalTok{,}
                 \AttributeTok{position =} \StringTok{"fill"}\NormalTok{)}
\end{Highlighting}
\end{Shaded}

\begin{verbatim}
## `stat_bin()` using `bins = 30`. Pick better value with `binwidth`.
\end{verbatim}

\begin{verbatim}
## Warning: Removed 10 rows containing missing values (geom_bar).
\end{verbatim}

\includegraphics{ADS502-Assignment-2.1-R_files/figure-latex/unnamed-chunk-1-6.pdf}

\begin{Shaded}
\begin{Highlighting}[]
\CommentTok{\# binned barchart}

\NormalTok{bank\_train}\SpecialCharTok{$}\NormalTok{duration\_binned }\OtherTok{\textless{}{-}} \FunctionTok{cut}\NormalTok{(}\AttributeTok{x =}\NormalTok{ bank\_train}\SpecialCharTok{$}\NormalTok{duration, }\AttributeTok{breaks =} \FunctionTok{c}\NormalTok{(}\DecValTok{0}\NormalTok{, }\DecValTok{1000}\NormalTok{, }\DecValTok{2000}\NormalTok{, }\DecValTok{3000}\NormalTok{,}\DecValTok{4000}\NormalTok{,}\DecValTok{5000}\NormalTok{),}
                             \AttributeTok{right =} \ConstantTok{FALSE}\NormalTok{,}
                             \AttributeTok{labels =} \FunctionTok{c}\NormalTok{(}\StringTok{"Under 1000"}\NormalTok{, }\StringTok{"1000 to 2000"}\NormalTok{, }\StringTok{"2000 to 3000"}\NormalTok{,}
                                        \StringTok{"3000 to 4000"}\NormalTok{, }\StringTok{"4000 to 5000"}\NormalTok{))}
\FunctionTok{ggplot}\NormalTok{(bank\_train, }\FunctionTok{aes}\NormalTok{(duration\_binned)) }\SpecialCharTok{+} \FunctionTok{geom\_bar}\NormalTok{(}\FunctionTok{aes}\NormalTok{(}\AttributeTok{fill =}\NormalTok{ response))}
\end{Highlighting}
\end{Shaded}

\includegraphics{ADS502-Assignment-2.1-R_files/figure-latex/unnamed-chunk-1-7.pdf}

\begin{Shaded}
\begin{Highlighting}[]
\FunctionTok{ggplot}\NormalTok{(bank\_train, }\FunctionTok{aes}\NormalTok{(duration\_binned)) }\SpecialCharTok{+} \FunctionTok{geom\_bar}\NormalTok{(}\FunctionTok{aes}\NormalTok{(}\AttributeTok{fill =}\NormalTok{ response), }\AttributeTok{position =} \StringTok{\textquotesingle{}fill\textquotesingle{}}\NormalTok{)}
\end{Highlighting}
\end{Shaded}

\includegraphics{ADS502-Assignment-2.1-R_files/figure-latex/unnamed-chunk-1-8.pdf}

\begin{Shaded}
\begin{Highlighting}[]
\CommentTok{\# For Exercises 14{-}20, work with the adult\_ch6\_training and adult\_ch6\_test data}
\CommentTok{\# sets. Use either Python or R to solve each problem.}

\CommentTok{\# 14. Create a CART model using the training data set that predicts income using}
\CommentTok{\# marital status and capital  gains and losses. Visualize the decision tree }
\CommentTok{\# (that is, provide the decision tree output). Describe the first few splits in the decision tree.}

\NormalTok{adult\_training }\OtherTok{\textless{}{-}} \FunctionTok{read.csv}\NormalTok{(}\AttributeTok{file =} \StringTok{"C:/Users/DDY/Desktop/2021{-}Spring{-}textbooks/ADS{-}502/Module2/Website Data Sets/adult\_ch6\_training"}\NormalTok{)}

\FunctionTok{colnames}\NormalTok{(adult\_training)[}\DecValTok{1}\NormalTok{] }\OtherTok{\textless{}{-}} \StringTok{"MaritalStatus"}

\NormalTok{adult\_training}\SpecialCharTok{$}\NormalTok{Income }\OtherTok{\textless{}{-}} \FunctionTok{factor}\NormalTok{(adult\_training}\SpecialCharTok{$}\NormalTok{Income)}
\NormalTok{adult\_training}\SpecialCharTok{$}\NormalTok{MaritalStatus }\OtherTok{\textless{}{-}} \FunctionTok{factor}\NormalTok{(adult\_training}\SpecialCharTok{$}\NormalTok{MaritalStatus)}


\FunctionTok{library}\NormalTok{(rpart); }\FunctionTok{library}\NormalTok{(rpart.plot)}

\NormalTok{DT\_CART }\OtherTok{\textless{}{-}} \FunctionTok{rpart}\NormalTok{(}\AttributeTok{formula =}\NormalTok{ Income }\SpecialCharTok{\textasciitilde{}}\NormalTok{ MaritalStatus }\SpecialCharTok{+}\NormalTok{ Cap\_Gains\_Losses,}\AttributeTok{data =} 
\NormalTok{                   adult\_training, }\AttributeTok{method =} \StringTok{"class"}\NormalTok{)}

\FunctionTok{rpart.plot}\NormalTok{(DT\_CART)}
\end{Highlighting}
\end{Shaded}

\includegraphics{ADS502-Assignment-2.1-R_files/figure-latex/unnamed-chunk-1-9.pdf}

\begin{Shaded}
\begin{Highlighting}[]
\NormalTok{?rpart.plot}
\end{Highlighting}
\end{Shaded}

\begin{verbatim}
## starting httpd help server ...
\end{verbatim}

\begin{verbatim}
##  done
\end{verbatim}

\begin{Shaded}
\begin{Highlighting}[]
\FunctionTok{rpart.plot}\NormalTok{(DT\_CART, }\AttributeTok{type =} \DecValTok{4}\NormalTok{, }\AttributeTok{extra =} \DecValTok{2}\NormalTok{)}
\end{Highlighting}
\end{Shaded}

\includegraphics{ADS502-Assignment-2.1-R_files/figure-latex/unnamed-chunk-1-10.pdf}

\begin{Shaded}
\begin{Highlighting}[]
\NormalTok{X }\OtherTok{=} \FunctionTok{data.frame}\NormalTok{(}\AttributeTok{MaritalStatus =}\NormalTok{ adult\_training}\SpecialCharTok{$}\NormalTok{MaritalStatus,}
               \AttributeTok{Cap\_Gains\_Losses =}
\NormalTok{                 adult\_training}\SpecialCharTok{$}\NormalTok{Cap\_Gains\_Losses)}

\NormalTok{predIncomeCART }\OtherTok{=} \FunctionTok{predict}\NormalTok{(}\AttributeTok{object =}\NormalTok{ DT\_CART, }\AttributeTok{newdata =}\NormalTok{ X,}
                         \AttributeTok{type =} \StringTok{"class"}\NormalTok{)}


\CommentTok{\# 15. Develop a CART model using the test data set that utilizes the same target }
\CommentTok{\# and predictor variables. Visualize the decision tree. Compare the decision trees. }
\CommentTok{\# Does the test data result match the training data result?}

\NormalTok{adult\_test }\OtherTok{\textless{}{-}} \FunctionTok{read.csv}\NormalTok{(}\AttributeTok{file =} \StringTok{"C:/Users/DDY/Desktop/2021{-}Spring{-}textbooks/ADS{-}502/Module2/Website Data Sets/adult\_ch6\_test"}\NormalTok{)}

\FunctionTok{colnames}\NormalTok{(adult\_test)[}\DecValTok{1}\NormalTok{] }\OtherTok{\textless{}{-}} \StringTok{"MaritalStatus"}
\NormalTok{adult\_test}\SpecialCharTok{$}\NormalTok{Income }\OtherTok{\textless{}{-}} \FunctionTok{factor}\NormalTok{(adult\_test}\SpecialCharTok{$}\NormalTok{Income)}
\NormalTok{adult\_test}\SpecialCharTok{$}\NormalTok{MaritalStatus }\OtherTok{\textless{}{-}} \FunctionTok{factor}\NormalTok{(adult\_test}\SpecialCharTok{$}\NormalTok{MaritalStatus)}
\NormalTok{DT\_CART\_test }\OtherTok{\textless{}{-}} \FunctionTok{rpart}\NormalTok{(}\AttributeTok{formula =}\NormalTok{ Income }\SpecialCharTok{\textasciitilde{}}\NormalTok{ MaritalStatus }\SpecialCharTok{+}\NormalTok{ Cap\_Gains\_Losses,}\AttributeTok{data =} 
\NormalTok{                   adult\_test, }\AttributeTok{method =} \StringTok{"class"}\NormalTok{)}

\FunctionTok{rpart.plot}\NormalTok{(DT\_CART\_test)}
\end{Highlighting}
\end{Shaded}

\includegraphics{ADS502-Assignment-2.1-R_files/figure-latex/unnamed-chunk-1-11.pdf}

\begin{Shaded}
\begin{Highlighting}[]
\NormalTok{?rpart.plot}

\FunctionTok{rpart.plot}\NormalTok{(DT\_CART\_test, }\AttributeTok{type =} \DecValTok{4}\NormalTok{, }\AttributeTok{extra =} \DecValTok{2}\NormalTok{)}
\end{Highlighting}
\end{Shaded}

\includegraphics{ADS502-Assignment-2.1-R_files/figure-latex/unnamed-chunk-1-12.pdf}

\begin{Shaded}
\begin{Highlighting}[]
\NormalTok{X\_test }\OtherTok{=} \FunctionTok{data.frame}\NormalTok{(}\AttributeTok{MaritalStatus =}\NormalTok{ adult\_test}\SpecialCharTok{$}\NormalTok{MaritalStatus,}
               \AttributeTok{Cap\_Gains\_Losses =}
\NormalTok{                 adult\_test}\SpecialCharTok{$}\NormalTok{Cap\_Gains\_Losses)}

\NormalTok{predIncomeCART\_test }\OtherTok{=} \FunctionTok{predict}\NormalTok{(}\AttributeTok{object =}\NormalTok{ DT\_CART\_test, }\AttributeTok{newdata =}\NormalTok{ X\_test,}
                         \AttributeTok{type =} \StringTok{"class"}\NormalTok{)}

\CommentTok{\# The decision tree of test dataset matches the one with training dataset.}


\CommentTok{\# 16. Use the training data set to build a C5.0 model to predict income using }
\CommentTok{\# marital status and capital gains and losses. Specify a minimum of 75 cases per}
\CommentTok{\# terminal node. Visualize the decision tree. Describe the first few splits in the decision tree.}

\FunctionTok{library}\NormalTok{(C50)}

\NormalTok{C5 }\OtherTok{\textless{}{-}} \FunctionTok{C5.0}\NormalTok{(}\AttributeTok{formula =}\NormalTok{ Income }\SpecialCharTok{\textasciitilde{}}\NormalTok{ MaritalStatus }\SpecialCharTok{+}\NormalTok{ Cap\_Gains\_Losses, }
           \AttributeTok{data =}\NormalTok{ adult\_training, }\AttributeTok{control =} \FunctionTok{C5.0Control}\NormalTok{(}\AttributeTok{minCases=}\DecValTok{75}\NormalTok{))}
\FunctionTok{plot}\NormalTok{(C5)}
\end{Highlighting}
\end{Shaded}

\includegraphics{ADS502-Assignment-2.1-R_files/figure-latex/unnamed-chunk-1-13.pdf}

\begin{Shaded}
\begin{Highlighting}[]
\CommentTok{\#predict(object = C5, newdata = X)}

\CommentTok{\# 17. How does your C5.0 model compare to the CART model? Describe the similarities and differences.}

\CommentTok{\# Similarities: Both CART and C50 follow the similar logic of test conditions;}
\CommentTok{\# Differences: CART starts the split with marital status and goes on with Cap\_Gains\_Losses}
\CommentTok{\# while c50 starts with Cap\_Gains\_Losses and goes on with marital status; Different }
\CommentTok{\# number of nodes and different ways of displaying classes for the leaf nodes.}


\CommentTok{\# For the following exercises, work with the bank\_reg\_training and the }
\CommentTok{\# bank\_reg\_test data sets. Use either Python or R to solve each problem.}

\CommentTok{\# 34. Use the training set to run a regression predicting Credit Score, }
\CommentTok{\# based on Debt{-}to{-}Income Ratio and Request Amount. Obtain a summary of the model.}
\CommentTok{\# Do both predictors belong in the model?}

\NormalTok{bank\_reg\_train }\OtherTok{=} \FunctionTok{read.csv}\NormalTok{(}\AttributeTok{file =}\StringTok{\textquotesingle{}C:/Users/DDY/Desktop/2021{-}Spring{-}textbooks/ADS{-}502/Module2/Website Data Sets/bank\_reg\_training\textquotesingle{}}\NormalTok{)}
\NormalTok{bank\_reg\_test }\OtherTok{=} \FunctionTok{read.csv}\NormalTok{(}\AttributeTok{file =}\StringTok{\textquotesingle{}C:/Users/DDY/Desktop/2021{-}Spring{-}textbooks/ADS{-}502/Module2/Website Data Sets/bank\_reg\_test\textquotesingle{}}\NormalTok{)}


\NormalTok{model01 }\OtherTok{\textless{}{-}} \FunctionTok{lm}\NormalTok{(}\AttributeTok{formula =}\NormalTok{ Credit.Score }\SpecialCharTok{\textasciitilde{}}\NormalTok{ Debt.to.Income.Ratio }\SpecialCharTok{+}\NormalTok{Request.Amount,}
              \AttributeTok{data =}\NormalTok{ bank\_reg\_train)}

\FunctionTok{summary}\NormalTok{(model01)}
\end{Highlighting}
\end{Shaded}

\begin{verbatim}
## 
## Call:
## lm(formula = Credit.Score ~ Debt.to.Income.Ratio + Request.Amount, 
##     data = bank_reg_train)
## 
## Residuals:
##     Min      1Q  Median      3Q     Max 
## -279.13  -25.11   10.87   39.93  175.32 
## 
## Coefficients:
##                        Estimate Std. Error t value Pr(>|t|)    
## (Intercept)           6.685e+02  1.336e+00  500.27   <2e-16 ***
## Debt.to.Income.Ratio -4.813e+01  4.785e+00  -10.06   <2e-16 ***
## Request.Amount        1.075e-03  6.838e-05   15.73   <2e-16 ***
## ---
## Signif. codes:  0 '***' 0.001 '**' 0.01 '*' 0.05 '.' 0.1 ' ' 1
## 
## Residual standard error: 66 on 10690 degrees of freedom
## Multiple R-squared:  0.02839,    Adjusted R-squared:  0.02821 
## F-statistic: 156.2 on 2 and 10690 DF,  p-value: < 2.2e-16
\end{verbatim}

\begin{Shaded}
\begin{Highlighting}[]
\CommentTok{\# 35. Validate the model from the previous exercise.}

\NormalTok{model02 }\OtherTok{\textless{}{-}} \FunctionTok{lm}\NormalTok{(}\AttributeTok{formula =}\NormalTok{ Credit.Score }\SpecialCharTok{\textasciitilde{}}\NormalTok{ Debt.to.Income.Ratio }\SpecialCharTok{+}\NormalTok{ Request.Amount,}
              \AttributeTok{data =}\NormalTok{ bank\_reg\_test)}

\FunctionTok{summary}\NormalTok{(model02)}
\end{Highlighting}
\end{Shaded}

\begin{verbatim}
## 
## Call:
## lm(formula = Credit.Score ~ Debt.to.Income.Ratio + Request.Amount, 
##     data = bank_reg_test)
## 
## Residuals:
##     Min      1Q  Median      3Q     Max 
## -288.16  -24.49   11.08   39.47  199.84 
## 
## Coefficients:
##                        Estimate Std. Error t value Pr(>|t|)    
## (Intercept)           6.655e+02  1.328e+00  501.26   <2e-16 ***
## Debt.to.Income.Ratio -5.214e+01  4.826e+00  -10.80   <2e-16 ***
## Request.Amount        1.302e-03  6.849e-05   19.01   <2e-16 ***
## ---
## Signif. codes:  0 '***' 0.001 '**' 0.01 '*' 0.05 '.' 0.1 ' ' 1
## 
## Residual standard error: 65.78 on 10772 degrees of freedom
## Multiple R-squared:  0.03845,    Adjusted R-squared:  0.03827 
## F-statistic: 215.4 on 2 and 10772 DF,  p-value: < 2.2e-16
\end{verbatim}

\begin{Shaded}
\begin{Highlighting}[]
\CommentTok{\# Validation complete.}

\CommentTok{\# 36. Use the regression equation to complete this sentence: "The estimated Credit Score equals.."}
\CommentTok{\# The estimated Credit Score equals y = 668.4562 {-} 48.1262* Debt{-}to{-}Income Ratio + 0.0011* Request Amount}

\CommentTok{\# 37. Interpret the coefficient for Debt{-}to{-}Income Ratio.}
\CommentTok{\# The coefficient for Debt{-}to{-}Income Ratio is negative which means the lower the}
\CommentTok{\# Debt{-}to{-}Income Ratio, the higher the credit score.}

\CommentTok{\# 38. Interpret the coefficient for Request Amount.}
\CommentTok{\# The coefficient for Request Amount is positive which means the higher the }
\CommentTok{\# Request Amount, the higher the credit score.}

\CommentTok{\# 39. Find and interpret the value of s.}
\CommentTok{\# Residual standard error: 65.78 on 10772 degrees of freedom. The size of model }
\CommentTok{\# prediction error is 65.8 (66), that is the difference between the actual }
\CommentTok{\# credit score and of which predicated from the model.}

\CommentTok{\# 40. Find and interpret Radj2 . Comment.}
\CommentTok{\# The adjusted R squared value is modified version of R{-}squared that has been }
\CommentTok{\# adjusted for the number of predictors in the model. It increases when the new }
\CommentTok{\# term improves the model more than would be expected by chance. It decreases }
\CommentTok{\# when a predictor improves the model by less than expected. The R{-}adj\^{}2 is 0.028 }
\CommentTok{\# from the model. This means that 2.8\% of the variability in Credit Score is }
\CommentTok{\# accounted for by the predictors Debt{-}to{-}Income Ratio and Request Amount.}

\CommentTok{\# 41. Find MAE\_Baseline and MAE\_Regression, and determine whether the regression }
\CommentTok{\# model outperformed its baseline model.}

\NormalTok{X\_test }\OtherTok{\textless{}{-}} \FunctionTok{data.frame}\NormalTok{(}\AttributeTok{Debt.to.Income.Ratio =}\NormalTok{ bank\_reg\_test}\SpecialCharTok{$}\NormalTok{Debt.to.Income.Ratio, }
                     \AttributeTok{Request.Amount =}\NormalTok{ bank\_reg\_test}\SpecialCharTok{$}\NormalTok{Request.Amount)}
\NormalTok{ypred }\OtherTok{\textless{}{-}} \FunctionTok{predict}\NormalTok{(}\AttributeTok{object =}\NormalTok{ model02, }\AttributeTok{newdata =}\NormalTok{ X\_test)}
\NormalTok{ytrue }\OtherTok{\textless{}{-}}\NormalTok{ bank\_reg\_test}\SpecialCharTok{$}\NormalTok{Credit.Score}

\FunctionTok{library}\NormalTok{(MLmetrics)}
\end{Highlighting}
\end{Shaded}

\begin{verbatim}
## 
## Attaching package: 'MLmetrics'
\end{verbatim}

\begin{verbatim}
## The following object is masked from 'package:base':
## 
##     Recall
\end{verbatim}

\begin{Shaded}
\begin{Highlighting}[]
\NormalTok{MAE\_Regression }\OtherTok{=} \FunctionTok{MAE}\NormalTok{(}\AttributeTok{y\_pred =}\NormalTok{ ypred, }\AttributeTok{y\_true =}\NormalTok{ ytrue)}

\NormalTok{y\_y\_bar }\OtherTok{=} \FunctionTok{abs}\NormalTok{(bank\_reg\_test}\SpecialCharTok{$}\NormalTok{Credit.Score }\SpecialCharTok{{-}} \FunctionTok{mean}\NormalTok{(bank\_reg\_test}\SpecialCharTok{$}\NormalTok{Credit.Score))}
\NormalTok{MAE\_Baseline }\OtherTok{=} \FunctionTok{sum}\NormalTok{(y\_y\_bar)}\SpecialCharTok{/}\FunctionTok{length}\NormalTok{(y\_y\_bar)}

\CommentTok{\# So the MAE\_Regression is 48.02 and the MAE\_Baseline is 48.60. }
\CommentTok{\# Since MAE\_Regression \textless{} MAE\_Baseline, thus, our regression model outperformed its baseline model.}
\end{Highlighting}
\end{Shaded}


\end{document}
